\documentclass[]{article}
\usepackage{lmodern}
\usepackage{amssymb,amsmath}
\usepackage{ifxetex,ifluatex}
\usepackage{fixltx2e} % provides \textsubscript
\ifnum 0\ifxetex 1\fi\ifluatex 1\fi=0 % if pdftex
  \usepackage[T1]{fontenc}
  \usepackage[utf8]{inputenc}
\else % if luatex or xelatex
  \ifxetex
    \usepackage{mathspec}
  \else
    \usepackage{fontspec}
  \fi
  \defaultfontfeatures{Ligatures=TeX,Scale=MatchLowercase}
\fi
% use upquote if available, for straight quotes in verbatim environments
\IfFileExists{upquote.sty}{\usepackage{upquote}}{}
% use microtype if available
\IfFileExists{microtype.sty}{%
\usepackage{microtype}
\UseMicrotypeSet[protrusion]{basicmath} % disable protrusion for tt fonts
}{}
\usepackage[margin=1in]{geometry}
\usepackage{hyperref}
\hypersetup{unicode=true,
            pdftitle={Unit3 HW},
            pdfauthor={Turner},
            pdfborder={0 0 0},
            breaklinks=true}
\urlstyle{same}  % don't use monospace font for urls
\usepackage{color}
\usepackage{fancyvrb}
\newcommand{\VerbBar}{|}
\newcommand{\VERB}{\Verb[commandchars=\\\{\}]}
\DefineVerbatimEnvironment{Highlighting}{Verbatim}{commandchars=\\\{\}}
% Add ',fontsize=\small' for more characters per line
\usepackage{framed}
\definecolor{shadecolor}{RGB}{248,248,248}
\newenvironment{Shaded}{\begin{snugshade}}{\end{snugshade}}
\newcommand{\AlertTok}[1]{\textcolor[rgb]{0.94,0.16,0.16}{#1}}
\newcommand{\AnnotationTok}[1]{\textcolor[rgb]{0.56,0.35,0.01}{\textbf{\textit{#1}}}}
\newcommand{\AttributeTok}[1]{\textcolor[rgb]{0.77,0.63,0.00}{#1}}
\newcommand{\BaseNTok}[1]{\textcolor[rgb]{0.00,0.00,0.81}{#1}}
\newcommand{\BuiltInTok}[1]{#1}
\newcommand{\CharTok}[1]{\textcolor[rgb]{0.31,0.60,0.02}{#1}}
\newcommand{\CommentTok}[1]{\textcolor[rgb]{0.56,0.35,0.01}{\textit{#1}}}
\newcommand{\CommentVarTok}[1]{\textcolor[rgb]{0.56,0.35,0.01}{\textbf{\textit{#1}}}}
\newcommand{\ConstantTok}[1]{\textcolor[rgb]{0.00,0.00,0.00}{#1}}
\newcommand{\ControlFlowTok}[1]{\textcolor[rgb]{0.13,0.29,0.53}{\textbf{#1}}}
\newcommand{\DataTypeTok}[1]{\textcolor[rgb]{0.13,0.29,0.53}{#1}}
\newcommand{\DecValTok}[1]{\textcolor[rgb]{0.00,0.00,0.81}{#1}}
\newcommand{\DocumentationTok}[1]{\textcolor[rgb]{0.56,0.35,0.01}{\textbf{\textit{#1}}}}
\newcommand{\ErrorTok}[1]{\textcolor[rgb]{0.64,0.00,0.00}{\textbf{#1}}}
\newcommand{\ExtensionTok}[1]{#1}
\newcommand{\FloatTok}[1]{\textcolor[rgb]{0.00,0.00,0.81}{#1}}
\newcommand{\FunctionTok}[1]{\textcolor[rgb]{0.00,0.00,0.00}{#1}}
\newcommand{\ImportTok}[1]{#1}
\newcommand{\InformationTok}[1]{\textcolor[rgb]{0.56,0.35,0.01}{\textbf{\textit{#1}}}}
\newcommand{\KeywordTok}[1]{\textcolor[rgb]{0.13,0.29,0.53}{\textbf{#1}}}
\newcommand{\NormalTok}[1]{#1}
\newcommand{\OperatorTok}[1]{\textcolor[rgb]{0.81,0.36,0.00}{\textbf{#1}}}
\newcommand{\OtherTok}[1]{\textcolor[rgb]{0.56,0.35,0.01}{#1}}
\newcommand{\PreprocessorTok}[1]{\textcolor[rgb]{0.56,0.35,0.01}{\textit{#1}}}
\newcommand{\RegionMarkerTok}[1]{#1}
\newcommand{\SpecialCharTok}[1]{\textcolor[rgb]{0.00,0.00,0.00}{#1}}
\newcommand{\SpecialStringTok}[1]{\textcolor[rgb]{0.31,0.60,0.02}{#1}}
\newcommand{\StringTok}[1]{\textcolor[rgb]{0.31,0.60,0.02}{#1}}
\newcommand{\VariableTok}[1]{\textcolor[rgb]{0.00,0.00,0.00}{#1}}
\newcommand{\VerbatimStringTok}[1]{\textcolor[rgb]{0.31,0.60,0.02}{#1}}
\newcommand{\WarningTok}[1]{\textcolor[rgb]{0.56,0.35,0.01}{\textbf{\textit{#1}}}}
\usepackage{graphicx,grffile}
\makeatletter
\def\maxwidth{\ifdim\Gin@nat@width>\linewidth\linewidth\else\Gin@nat@width\fi}
\def\maxheight{\ifdim\Gin@nat@height>\textheight\textheight\else\Gin@nat@height\fi}
\makeatother
% Scale images if necessary, so that they will not overflow the page
% margins by default, and it is still possible to overwrite the defaults
% using explicit options in \includegraphics[width, height, ...]{}
\setkeys{Gin}{width=\maxwidth,height=\maxheight,keepaspectratio}
\IfFileExists{parskip.sty}{%
\usepackage{parskip}
}{% else
\setlength{\parindent}{0pt}
\setlength{\parskip}{6pt plus 2pt minus 1pt}
}
\setlength{\emergencystretch}{3em}  % prevent overfull lines
\providecommand{\tightlist}{%
  \setlength{\itemsep}{0pt}\setlength{\parskip}{0pt}}
\setcounter{secnumdepth}{0}
% Redefines (sub)paragraphs to behave more like sections
\ifx\paragraph\undefined\else
\let\oldparagraph\paragraph
\renewcommand{\paragraph}[1]{\oldparagraph{#1}\mbox{}}
\fi
\ifx\subparagraph\undefined\else
\let\oldsubparagraph\subparagraph
\renewcommand{\subparagraph}[1]{\oldsubparagraph{#1}\mbox{}}
\fi

%%% Use protect on footnotes to avoid problems with footnotes in titles
\let\rmarkdownfootnote\footnote%
\def\footnote{\protect\rmarkdownfootnote}

%%% Change title format to be more compact
\usepackage{titling}

% Create subtitle command for use in maketitle
\providecommand{\subtitle}[1]{
  \posttitle{
    \begin{center}\large#1\end{center}
    }
}

\setlength{\droptitle}{-2em}

  \title{Unit3 HW}
    \pretitle{\vspace{\droptitle}\centering\huge}
  \posttitle{\par}
    \author{Turner}
    \preauthor{\centering\large\emph}
  \postauthor{\par}
      \predate{\centering\large\emph}
  \postdate{\par}
    \date{12/16/2018}


\begin{document}
\maketitle

\hypertarget{hw-instructions}{%
\subsection{HW Instructions}\label{hw-instructions}}

The weekly HW assignments are designed to accomplish 2 goals for the
MSDS student. The first is to provide a series of conceptual and
analtical questions so the student can get a feel for their current
understanding of the unit. The second goal is to introduce the students
to standard functions and routines in R that effectively do the same
things that the ``Procs'' do in SAS.

R and SAS are both wonderful tools and as we go through the assignments,
students will begin to recognize very quickly that they both have pros
and cons.

The formatting of the HW is as follows:\\
1. A series of high level questions will be asked with either short
answers or simple multiple choice responses.\\
2. Analytical questions will be provided but a short vignette example of
how R functions work for a given topic or method will be given. The
student will then be asked a follow up question or two based on the
output provided.\\
3. Thirdly, a new data set will be given to allow the student to gain
some experience with a new data set from start to finish.

Solutions to the HW will be provided a day or two after the HW is
submitted. It is up to the student to ``shore up'' any confusion or
missunderstanding of a topic. Grading will be based on a combination of
correctness, completion, and overall conciseness.

The student may provide there answers in a seperate word document. Just
make sure that it is easy to follow and that all questions have been
addressed for the grader. You are welcome to use R markdown, but it is
not required.

\hypertarget{anova-conceptual-questions}{%
\subsection{ANOVA Conceptual
questions}\label{anova-conceptual-questions}}

\begin{enumerate}
\def\labelenumi{\arabic{enumi}.}
\item
  State the necessary assumptions for Two Way ANOVA analysis to
  beconducted. Note: That addative versus non additivie is not a
  component of the assumptions
\item
  State in words what it means for there to be an ``interaction''
  between two explanatory variables. Note: Do not explain the meaning in
  terms of a graph with parallel lines.
\item
  What is the family wise error rate? What is multiple testing and why
  is it an issue when conducting ANOVA type models such as Two Way
  ANOVA?
\item
  True or False? The overall Type-III sums of squares F-test's allow the
  analyst to determine where specific differences lie between levels of
  the factor.
\end{enumerate}

\hypertarget{exercise-1-act-scores-revisited}{%
\subsection{Exercise \#1 ACT Scores
Revisited}\label{exercise-1-act-scores-revisited}}

The first step in any analysis is appropriately describing the data both
numerically and visually. For a Two Way ANOVA analysis, one of the most
helpful visual tools is the mean profile plot (with or without the raw
data). The following code reads in the ACT data set from our pre live
discussion and provides a handy, modifiable, function that can make a
quick summary statistics table really quick.

\begin{Shaded}
\begin{Highlighting}[]
\KeywordTok{setwd}\NormalTok{(}\StringTok{"~/Desktop/MSDS_NEW/ZippedFiles/Unit3/Unit3PreLive"}\NormalTok{)}

\NormalTok{ACT<-}\KeywordTok{read.csv}\NormalTok{(}\StringTok{"MathACT_2.csv"}\NormalTok{)}


\CommentTok{#Attaching the data set, creating a function, and creating a summary stats table.  Note: In line 44 below, you can add other statistics like median, IQR,etc.}

\KeywordTok{attach}\NormalTok{(ACT)}
\NormalTok{mysummary<-}\ControlFlowTok{function}\NormalTok{(x)\{}
\NormalTok{  result<-}\KeywordTok{c}\NormalTok{(}\KeywordTok{length}\NormalTok{(x),}\KeywordTok{mean}\NormalTok{(x),}\KeywordTok{sd}\NormalTok{(x),}\KeywordTok{sd}\NormalTok{(x)}\OperatorTok{/}\KeywordTok{length}\NormalTok{(x))}
  \KeywordTok{names}\NormalTok{(result)<-}\KeywordTok{c}\NormalTok{(}\StringTok{"N"}\NormalTok{,}\StringTok{"Mean"}\NormalTok{,}\StringTok{"SD"}\NormalTok{,}\StringTok{"SE"}\NormalTok{)}
  \KeywordTok{return}\NormalTok{(result)}
\NormalTok{\}}
\NormalTok{sumstats<-}\KeywordTok{aggregate}\NormalTok{(Score}\OperatorTok{~}\NormalTok{Background}\OperatorTok{*}\NormalTok{Sex,}\DataTypeTok{data=}\NormalTok{ACT,mysummary)}
\NormalTok{sumstats<-}\KeywordTok{cbind}\NormalTok{(sumstats[,}\DecValTok{1}\OperatorTok{:}\DecValTok{2}\NormalTok{],sumstats[,}\OperatorTok{-}\NormalTok{(}\DecValTok{1}\OperatorTok{:}\DecValTok{2}\NormalTok{)])}
\NormalTok{sumstats}
\end{Highlighting}
\end{Shaded}

\begin{verbatim}
##   Background    Sex   N      Mean       SD         SE
## 1          a female  82  9.073171 4.186340 0.05105293
## 2          b female 387 13.963824 5.000905 0.01292224
## 3          c female  54 22.629630 4.849806 0.08981122
## 4          a   male  48 11.458333 5.086312 0.10596483
## 5          b   male 223 18.565022 4.888305 0.02192065
## 6          c   male  67 32.432836 5.554752 0.08290675
\end{verbatim}

With the three levels of background and two levels of sex status, the
table provides the sample size, mean, standard deviation, and the means
standard error for each of the 6 combinations of the two factors
combined. This can be used to take a quick look at the data to see if
things are making sense. Adding additional summaries like the max, min,
and quartiles would be heplful as well.

The above table may not be too aesthetically pleasing. Luckily under the
current format of the table, its quite easy to generate a means profile
plot to visualize the data. This graphic was most likely a major point
of discussion during live session.

\begin{Shaded}
\begin{Highlighting}[]
\KeywordTok{library}\NormalTok{(ggplot2)}
\KeywordTok{ggplot}\NormalTok{(sumstats,}\KeywordTok{aes}\NormalTok{(}\DataTypeTok{x=}\NormalTok{Background,}\DataTypeTok{y=}\NormalTok{Mean,}\DataTypeTok{group=}\NormalTok{Sex,}\DataTypeTok{colour=}\NormalTok{Sex))}\OperatorTok{+}
\StringTok{  }\KeywordTok{ylab}\NormalTok{(}\StringTok{"ACT Score"}\NormalTok{)}\OperatorTok{+}
\StringTok{  }\KeywordTok{geom_line}\NormalTok{()}\OperatorTok{+}
\StringTok{  }\KeywordTok{geom_point}\NormalTok{()}\OperatorTok{+}
\StringTok{  }\KeywordTok{geom_errorbar}\NormalTok{(}\KeywordTok{aes}\NormalTok{(}\DataTypeTok{ymin=}\NormalTok{Mean}\OperatorTok{-}\NormalTok{SE,}\DataTypeTok{ymax=}\NormalTok{Mean}\OperatorTok{+}\NormalTok{SE),}\DataTypeTok{width=}\NormalTok{.}\DecValTok{1}\NormalTok{)}
\end{Highlighting}
\end{Shaded}

\includegraphics{Unit3_HW_files/figure-latex/meanplot-1.pdf}

\textbf{HOMEWORK QUESTION}

\begin{enumerate}
\def\labelenumi{\arabic{enumi}.}
\item
  Modify the previous R script so that the summary table also includeds
  the min, the max, and IQR. These functions are all self
  explanatory\ldots{}min(x), max(x), IQR(x).
\item
  Create another means plot but rather than using the standard errors
  (SE) to make the error bars. Make it with the raw standard deviations
  (SD). Which graphic (compared to plot using SE) is more telling about
  the assumption of equal variances for the ANOVA model? Give a little
  explanation for your answer.
\end{enumerate}

\hypertarget{exercise-2-conducting-a-two-way-anova-analysis-in-r}{%
\subsection{Exercise \#2 Conducting a Two Way ANOVA Analysis in
R}\label{exercise-2-conducting-a-two-way-anova-analysis-in-r}}

Since Two Way ANOVA's are techically just special cases of multiple
linear regression, it's not to suprising that the same function call is
used to build the model. After viewing and exploring the data via
Exercise 1. The next step would be to fit a full nonaddative model,
check the assumptions of the model, and then examine the type III sums
of squares F tables.

The following code fits the nonadditive two way anova model and then
produces the first the main residual diagnostics for assumption
checking. The syntax for including interaction terms is slightly
different so please make note.

\begin{Shaded}
\begin{Highlighting}[]
\NormalTok{model.fit<-}\KeywordTok{aov}\NormalTok{(Score}\OperatorTok{~}\NormalTok{Background}\OperatorTok{+}\NormalTok{Sex}\OperatorTok{+}\NormalTok{Background}\OperatorTok{:}\NormalTok{Sex,}\DataTypeTok{data=}\NormalTok{ACT)}
\KeywordTok{par}\NormalTok{(}\DataTypeTok{mfrow=}\KeywordTok{c}\NormalTok{(}\DecValTok{1}\NormalTok{,}\DecValTok{2}\NormalTok{))}
\KeywordTok{plot}\NormalTok{(model.fit}\OperatorTok{$}\NormalTok{fitted.values,model.fit}\OperatorTok{$}\NormalTok{residuals,}\DataTypeTok{ylab=}\StringTok{"Resdiduals"}\NormalTok{,}\DataTypeTok{xlab=}\StringTok{"Fitted"}\NormalTok{)}
\KeywordTok{qqnorm}\NormalTok{(model.fit}\OperatorTok{$}\NormalTok{residuals)}
\end{Highlighting}
\end{Shaded}

\includegraphics{Unit3_HW_files/figure-latex/modelfit-1.pdf}

The previous graphics are not very pretty. We can use the ggplot2
package to jazz things up a bit.

\begin{Shaded}
\begin{Highlighting}[]
\KeywordTok{library}\NormalTok{(gridExtra)}
\NormalTok{myfits<-}\KeywordTok{data.frame}\NormalTok{(}\DataTypeTok{fitted.values=}\NormalTok{model.fit}\OperatorTok{$}\NormalTok{fitted.values,}\DataTypeTok{residuals=}\NormalTok{model.fit}\OperatorTok{$}\NormalTok{residuals)}

\CommentTok{#Residual vs Fitted}
\NormalTok{plot1<-}\KeywordTok{ggplot}\NormalTok{(myfits,}\KeywordTok{aes}\NormalTok{(}\DataTypeTok{x=}\NormalTok{fitted.values,}\DataTypeTok{y=}\NormalTok{residuals))}\OperatorTok{+}\KeywordTok{ylab}\NormalTok{(}\StringTok{"Residuals"}\NormalTok{)}\OperatorTok{+}
\StringTok{  }\KeywordTok{xlab}\NormalTok{(}\StringTok{"Predicted"}\NormalTok{)}\OperatorTok{+}\KeywordTok{geom_point}\NormalTok{()}

\CommentTok{#QQ plot of residuals  #Note the diagonal abline is only good for qqplots of normal data.}
\NormalTok{plot2<-}\KeywordTok{ggplot}\NormalTok{(myfits,}\KeywordTok{aes}\NormalTok{(}\DataTypeTok{sample=}\NormalTok{residuals))}\OperatorTok{+}
\StringTok{  }\KeywordTok{stat_qq}\NormalTok{()}\OperatorTok{+}\KeywordTok{geom_abline}\NormalTok{(}\DataTypeTok{intercept=}\KeywordTok{mean}\NormalTok{(myfits}\OperatorTok{$}\NormalTok{residuals), }\DataTypeTok{slope =} \KeywordTok{sd}\NormalTok{(myfits}\OperatorTok{$}\NormalTok{residuals))}

\CommentTok{#Histogram of residuals}
\NormalTok{plot3<-}\KeywordTok{ggplot}\NormalTok{(myfits, }\KeywordTok{aes}\NormalTok{(}\DataTypeTok{x=}\NormalTok{residuals)) }\OperatorTok{+}\StringTok{ }
\StringTok{  }\KeywordTok{geom_histogram}\NormalTok{(}\KeywordTok{aes}\NormalTok{(}\DataTypeTok{y=}\NormalTok{..density..),}\DataTypeTok{binwidth=}\DecValTok{1}\NormalTok{,}\DataTypeTok{color=}\StringTok{"black"}\NormalTok{, }\DataTypeTok{fill=}\StringTok{"gray"}\NormalTok{)}\OperatorTok{+}
\StringTok{  }\KeywordTok{geom_density}\NormalTok{(}\DataTypeTok{alpha=}\NormalTok{.}\DecValTok{1}\NormalTok{, }\DataTypeTok{fill=}\StringTok{"red"}\NormalTok{)}

\KeywordTok{grid.arrange}\NormalTok{(plot1, plot2,plot3, }\DataTypeTok{ncol=}\DecValTok{3}\NormalTok{)}
\end{Highlighting}
\end{Shaded}

\includegraphics{Unit3_HW_files/figure-latex/unnamed-chunk-1-1.pdf}

As discussed in class, the residual diagnostics do not provide any
concern about the assumptions of a two way anova analysis. If there
were, we would have to address those concerns via a transformation of
the response or multiple analysis with and without outliers, etc.
Examining the type-III sums of squares F table we have:

\begin{Shaded}
\begin{Highlighting}[]
\KeywordTok{library}\NormalTok{(car)}
\end{Highlighting}
\end{Shaded}

\begin{verbatim}
## Loading required package: carData
\end{verbatim}

\begin{Shaded}
\begin{Highlighting}[]
\KeywordTok{Anova}\NormalTok{(model.fit,}\DataTypeTok{type=}\DecValTok{3}\NormalTok{)}
\end{Highlighting}
\end{Shaded}

\begin{verbatim}
## Anova Table (Type III tests)
## 
## Response: Score
##                 Sum Sq  Df  F value    Pr(>F)    
## (Intercept)     6750.4   1 276.4610 < 2.2e-16 ***
## Background      6007.9   2 123.0263 < 2.2e-16 ***
## Sex              172.2   1   7.0542  0.008055 ** 
## Background:Sex   919.0   2  18.8192 1.004e-08 ***
## Residuals      20876.8 855                       
## ---
## Signif. codes:  0 '***' 0.001 '**' 0.01 '*' 0.05 '.' 0.1 ' ' 1
\end{verbatim}

Writing contrasts are a little more cumbersome in R. To help you guys
out and alleviate the need to keep track of all of the zero's and one's,
I've wrote a little script that allows you to just specify the contrast
that you want in a slightly simpler way. But first lets use some tools
that provides a blanket lists of comparisons. Since there is no
significant interaction, we just need to examine each factor one at a
time. To examine all pairwise comparisons for say ``background'', the
following script provides the t-test results adjusted for multiple tests
using Tukey's procedure.

\begin{Shaded}
\begin{Highlighting}[]
\KeywordTok{TukeyHSD}\NormalTok{(model.fit,}\StringTok{"Background"}\NormalTok{,}\DataTypeTok{conf.level=}\NormalTok{.}\DecValTok{95}\NormalTok{)}
\end{Highlighting}
\end{Shaded}

\begin{verbatim}
##   Tukey multiple comparisons of means
##     95% family-wise confidence level
## 
## Fit: aov(formula = Score ~ Background + Sex + Background:Sex, data = ACT)
## 
## $Background
##          diff       lwr       upr p adj
## b-a  5.692055  4.571357  6.812754     0
## c-a 18.104005 16.638518 19.569492     0
## c-b 12.411950 11.257405 13.566494     0
\end{verbatim}

The table is helpful for quickly examining the results and getting the
p-values and estimates. Its always helpful to visualize.

\begin{Shaded}
\begin{Highlighting}[]
\KeywordTok{plot}\NormalTok{(}\KeywordTok{TukeyHSD}\NormalTok{(model.fit,}\StringTok{"Background"}\NormalTok{,}\DataTypeTok{conf.level=}\NormalTok{.}\DecValTok{95}\NormalTok{))}
\end{Highlighting}
\end{Shaded}

\includegraphics{Unit3_HW_files/figure-latex/unnamed-chunk-4-1.pdf}

If an interaction is present, you can rinse and repeat the code just
using the interaction term instead. This code below is for illustration,
it makes no sense to do this on the ACT data set since the interaction F
test is not significant.

\begin{Shaded}
\begin{Highlighting}[]
\KeywordTok{TukeyHSD}\NormalTok{(model.fit,}\StringTok{"Background:Sex"}\NormalTok{,}\DataTypeTok{conf.level=}\NormalTok{.}\DecValTok{95}\NormalTok{)}
\end{Highlighting}
\end{Shaded}

\begin{verbatim}
##   Tukey multiple comparisons of means
##     95% family-wise confidence level
## 
## Fit: aov(formula = Score ~ Background + Sex + Background:Sex, data = ACT)
## 
## $`Background:Sex`
##                         diff         lwr       upr     p adj
## b:female-a:female   4.890654   3.1748785  6.606429 0.0000000
## c:female-a:female  13.556459  11.0830154 16.029902 0.0000000
## a:male-a:female     2.385163  -0.1797967  4.950122 0.0854058
## b:male-a:female     9.491852   7.6691025 11.314601 0.0000000
## c:male-a:female    23.359665  21.0354027 25.683927 0.0000000
## c:female-b:female   8.665805   6.6155720 10.716039 0.0000000
## a:male-b:female    -2.505491  -4.6652479 -0.345734 0.0123022
## b:male-b:female     4.601198   3.4146282  5.787768 0.0000000
## c:male-b:female    18.469012  16.6014652 20.336558 0.0000000
## a:male-c:female   -11.171296 -13.9710443 -8.371548 0.0000000
## b:male-c:female    -4.064607  -6.2051648 -1.924050 0.0000011
## c:male-c:female     9.803206   7.2221661 12.384246 0.0000000
## b:male-a:male       7.106689   4.8610087  9.352369 0.0000000
## c:male-a:male      20.974502  18.3056335 23.643371 0.0000000
## c:male-b:male      13.867813  11.9015327 15.834094 0.0000000
\end{verbatim}

\begin{Shaded}
\begin{Highlighting}[]
\KeywordTok{plot}\NormalTok{(}\KeywordTok{TukeyHSD}\NormalTok{(model.fit,}\StringTok{"Background:Sex"}\NormalTok{,}\DataTypeTok{conf.level=}\NormalTok{.}\DecValTok{95}\NormalTok{))}
\end{Highlighting}
\end{Shaded}

\includegraphics{Unit3_HW_files/figure-latex/unnamed-chunk-5-1.pdf}

As discussed in class, including all possible combinations of
comparisons may be too much and of little interest to the actual study
at hand. We can manually create the comparisons of interest and manual
adjust the p-values through writing contrasts. To help streamline this
for you guys, I've included a little R script that makes the process a
little more automated for you.

The following script allow you to write out your contrasts in a more
verbal syntax. I'll run you through the most tedious scenario. The
script can be easily modified to handle simpler situations. First things
first, all you need to do is provide some details as to what comparisons
you'd like to make. Suppose, that if the interaction was significant,
the only meaningful comparisons to make in the analysis comparing males
versus females for each level of background.

\begin{Shaded}
\begin{Highlighting}[]
\KeywordTok{library}\NormalTok{(lsmeans) }\CommentTok{#maybe need eemeans package}
\end{Highlighting}
\end{Shaded}

\begin{verbatim}
## Loading required package: emmeans
\end{verbatim}

\begin{verbatim}
## The 'lsmeans' package is now basically a front end for 'emmeans'.
## Users are encouraged to switch the rest of the way.
## See help('transition') for more information, including how to
## convert old 'lsmeans' objects and scripts to work with 'emmeans'.
\end{verbatim}

\begin{Shaded}
\begin{Highlighting}[]
\NormalTok{contrast.factor<-}\ErrorTok{~}\NormalTok{Background}\OperatorTok{*}\NormalTok{Sex}
\NormalTok{mycontrast<-}\KeywordTok{c}\NormalTok{(}\StringTok{"amale-afemale"}\NormalTok{,}\StringTok{"bmale-bfemale"}\NormalTok{,}\StringTok{"cmale-cfemale"}\NormalTok{)}
\NormalTok{dat<-ACT}
\end{Highlighting}
\end{Shaded}

The above piece of code provides no output, but formats things for the
following code to run. The key player here is the ``contrast.factor''
and the ``mycontrast'' objects. The contrast.factor piece is just
specifiying what types of comparisons you would like to make. For
example, if we only wanted to compare the background levels we would
have just specified ``\textasciitilde{}Background''. The ``mycontrast''
object is where you get to specify what comparisons you would like to
make. For a single factor, you just simply write out the factor levels
you want to compare with a subtration between them. For an interaction
type comparison the syntax depends on what was used in the
contrast.factor object. In our example, background is listed first, so
when making comparisons the levels of background are concatenated to the
levels of Sex before subtracting which combinations you want to compare.

The following code is something I wrote that takes the information you
specified above and creates a clean table of resutls with bonferroni
adjusted p-values. This script can be reused over and over, just
changing the initial starting script is all that is required.

\begin{Shaded}
\begin{Highlighting}[]
\CommentTok{#Running a loop that determines the appropriate 0's and 1's for each }
\CommentTok{#contrast specified above.}
\KeywordTok{library}\NormalTok{(limma)}
\NormalTok{final.result<-}\KeywordTok{c}\NormalTok{()}
\ControlFlowTok{for}\NormalTok{( j }\ControlFlowTok{in} \DecValTok{1}\OperatorTok{:}\KeywordTok{length}\NormalTok{(mycontrast))\{}
\NormalTok{contrast.factor.names<-}\KeywordTok{gsub}\NormalTok{(}\StringTok{" "}\NormalTok{, }\StringTok{""}\NormalTok{, }\KeywordTok{unlist}\NormalTok{(}\KeywordTok{strsplit}\NormalTok{(}\KeywordTok{as.character}\NormalTok{(contrast.factor),}\DataTypeTok{split =} \StringTok{"*"}\NormalTok{, }\DataTypeTok{fixed =}\NormalTok{ T))[}\OperatorTok{-}\DecValTok{1}\NormalTok{])}
\NormalTok{contrast.factor}\FloatTok{.2}\NormalTok{ <-}\StringTok{ }\KeywordTok{vector}\NormalTok{(}\StringTok{"list"}\NormalTok{, }\KeywordTok{length}\NormalTok{(contrast.factor.names))}
\ControlFlowTok{for}\NormalTok{ (i }\ControlFlowTok{in} \DecValTok{1}\OperatorTok{:}\KeywordTok{length}\NormalTok{(contrast.factor.names)) \{}
\NormalTok{  contrast.factor}\FloatTok{.2}\NormalTok{[[i]] <-}\StringTok{ }\KeywordTok{levels}\NormalTok{(dat[, contrast.factor.names[i]])}
\NormalTok{\}}
\NormalTok{new.factor.levels <-}\StringTok{ }\KeywordTok{do.call}\NormalTok{(paste, }\KeywordTok{c}\NormalTok{(}\KeywordTok{do.call}\NormalTok{(expand.grid, }
\NormalTok{                                              contrast.factor}\FloatTok{.2}\NormalTok{), }\DataTypeTok{sep =} \StringTok{""}\NormalTok{))}
\NormalTok{temp.cont<-mycontrast[j]}
\NormalTok{contrast2 <-}\StringTok{ }\KeywordTok{list}\NormalTok{(}\DataTypeTok{comparison =} \KeywordTok{as.vector}\NormalTok{(}\KeywordTok{do.call}\NormalTok{(makeContrasts, }
                                                \KeywordTok{list}\NormalTok{(}\DataTypeTok{contrasts =}\NormalTok{ temp.cont, }\DataTypeTok{levels =}\NormalTok{ new.factor.levels))))}

\NormalTok{contrast.result <-}\StringTok{ }\KeywordTok{summary}\NormalTok{(}\KeywordTok{contrast}\NormalTok{(}\KeywordTok{lsmeans}\NormalTok{(model.fit, }
\NormalTok{                                            contrast.factor), contrast2, }\DataTypeTok{by =} \OtherTok{NULL}\NormalTok{))}

\NormalTok{final.result<-}\KeywordTok{rbind}\NormalTok{(final.result,contrast.result)}
\NormalTok{\}}
\CommentTok{#Cleaning up and applying bonferroni correction to the number}
\CommentTok{#of total comparisons investigated.}
\NormalTok{final.result}\OperatorTok{$}\NormalTok{contrast<-mycontrast}
\NormalTok{final.result}\OperatorTok{$}\NormalTok{bonf<-}\KeywordTok{length}\NormalTok{(mycontrast)}\OperatorTok{*}\NormalTok{final.result}\OperatorTok{$}\NormalTok{p.value}
\NormalTok{final.result}\OperatorTok{$}\NormalTok{bonf[final.result}\OperatorTok{$}\NormalTok{bonf}\OperatorTok{>}\DecValTok{1}\NormalTok{]<-}\DecValTok{1}

\NormalTok{final.result}
\end{Highlighting}
\end{Shaded}

\begin{verbatim}
##  contrast      estimate    SE  df t.ratio p.value   bonf
##  amale-afemale     2.39 0.898 855  2.656  0.0081  0.0242
##  bmale-bfemale     4.60 0.415 855 11.076  <.0001  0.0000
##  cmale-cfemale     9.80 0.904 855 10.848  <.0001  0.0000
\end{verbatim}

\textbf{HOMEWORK QUESTION}

\begin{enumerate}
\def\labelenumi{\arabic{enumi}.}
\tightlist
\item
  Consider comparing the mean ACT scores of males versus females
  specifically for background A. Compare the outputs from the Tukey
  comparison result table to that of the output generated from my manual
  contrast maker. Is the estimated differences the same? Can you explain
  why are the adjusted p-values different for the two result tables? One
  would suggest that we reject the null while the other would have us to
  fail to reject. (This is just a conceptual thinking question. The
  interaction term is not significant for this data analysis.)
\end{enumerate}

\hypertarget{exercise-3}{%
\subsection{Exercise \#3}\label{exercise-3}}

Lets examine the dta Exercise 13.17 from the statistical sleuth book.
The data set is easily accesable in R via the following package.

\begin{Shaded}
\begin{Highlighting}[]
\KeywordTok{library}\NormalTok{(Sleuth3)}
\KeywordTok{head}\NormalTok{(ex1317)}
\end{Highlighting}
\end{Shaded}

\begin{verbatim}
##   Iridium    Strata DepthCat
## 1      75 Limestone        1
## 2     200 Limestone        1
## 3     120 Limestone        2
## 4     310 Limestone        2
## 5     290 Limestone        3
## 6     450 Limestone        3
\end{verbatim}

\begin{enumerate}
\def\labelenumi{\arabic{enumi}.}
\item
  Provide a means plot of the data. Use this along with any additional
  information to comment on whether an addative or nonadditive model is
  probably the most appropriated. If it is not obvious that is okay just
  do your best.
\item
  Fit a nonadditive 2 way anova model to the data set and provide the
  residual diagnostics. Comment on the appropriateness of the current
  anova fit.
\item
  Provide the type 3 ANOVA F-tests. Answer the following question using
  the table. Do the potential changes in mean Iridium by strata depend
  on the depth?
\item
  Using multple testing techniques, determine what factors (or
  combinations) contribute to changes in mean iridium.
\end{enumerate}


\end{document}
